\documentclass{article} % For LaTeX2e
\usepackage{nips15submit_e,times}
\usepackage{hyperref}
\usepackage{url}
\usepackage[pdftex]{graphicx}
\usepackage{float}
\usepackage{amsmath}
%\documentstyle[nips14submit_09,times,art10]{article} % For LaTeX 2.09


\title{Information Retrieval and DataMining:\\
		Application of Neural Networks to Time Series
}


\author{
Mark Neumann, Alexander Chapovskiy and Zheng Tian\\ %thanks{} \\
UCL\\
\texttt{mark.neumann.15@ucl.ac.uk} \\
\texttt{alexander.chapovskiy.15@ucl.ac.uk} \\
\texttt{zheng.tian.11@ucl.ac.uk@ucl.ac.uk} \\
}
\newcommand{\fix}{\marginpar{FIX}}
\newcommand{\new}{\marginpar{NEW}}
\newcommand{\be}{\begin{equation}}
\newcommand{\ee}{\end{equation}}
\newcommand{\MSEtest}{MSE_{\rm \scriptsize test}}
%\DeclareMathOperator*{\tanh}{tanh}


\nipsfinalcopy % Uncomment for camera-ready version

\begin{document}


\maketitle
\graphicspath{{images/}}

\begin{abstract}
This is a report for group project in the Information Retrieval and Data 
Mining course. In this report we investigate different deep neural network
architectures on the time series. In particular, we apply neural 
networks to energy load and household energy consumption datasets.
\end{abstract}


\section{Introduction}
\label{sec:intro}

\section{Datasets}
\label{sec:data}

\subsection{Energy Load}
\label{sec:data/energy}

!!!Description of the dataset

This dataset was used in the Kaggle competition. After the competition
the winners of the competition published their approaches, 
\cite{energy_kaggle}.
Mostly the approach is linear regression with non-linear features.

Neural networks are pplied to this dataset in \cite{enery_nn}

!!!! Before we apply out models to the dataset we perform the 
following preprocessing.\\

\subsection{Household Energy Consumption}
\label{sec:data/house}

\section{Regression with Non-linear Features}
\label{sec:reg}

\section{Reccurent Neural Networks}
\label{sec:nn}


\subsection{Reccurent Neural Network}
\label{sec:nn/rnn}

\subsection{LSTM}
\label{sec:nn/lstm}


\section{Conclusions}




\begin{thebibliography}{9}

\small{


\bibitem{energy_kaggle}
Tao Hong, Pierre Pinson and Shu Fan, 
(2014), 
{\it Global Energy Forecasting Competition 2012.}
International Journal of Forecasting, 30(2), 357;
\\
Charlton, N. and Singleton, C. 
(2014). 
{\it A refined parametric model for short term load forecasting.}
International Journal of Forecasting,
30(2), 364-368.
\\
Souhaib Ben Taieba and Rob J. Hyndmanb,
(2014), 
{\it A gradient boosting approach to the Kaggle load forecasting 
competition.}
International Journal of Forecasting, 30(2), 382;
\\
Lloyd, J. R. (2014), 
{\it GEFCom2012 hierarchical load forecasting: Gradient boosting machines and Gaussian processes. }
International Journal of Forecasting, 30(2), 369.


}

\end{thebibliography}

\end{document}
